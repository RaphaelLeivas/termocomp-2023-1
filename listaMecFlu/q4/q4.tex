
\section{Quarta Questão (9 pts)}

\subsection{Análise Analítica}

\numberwithin{equation}{section}
\numberwithin{figure}{section}

De modo parecido à modelagem do Prolema 3, temos que partir das equações de Navier-Stokes e 
da equação de conservação da massa, assumindo escoamento incompressível. 
Podemos fazer as seguintes simplificações:

\begin{itemize}
    \item Estamos modelando a geometria apenas em duas dimensões $x$ e $y$: $w = 0$ e derivadas em relação a $z$ nulas.
    \item Gravidade atua apenas na componente $y$: $g_x = g_z = 0$
    \item Regime permanente: derivadas em relação a $t$ nulas
    $\diffp{u}{z} = \diffp[2]{u}{z} = 0$ 
\end{itemize}

Com essas simplificações, temos três equações que modelam o problema

\begin{equation}\label{eq:Q4NS1}
        \rho\left(u\diffp{u}{x} + v\diffp{u}{y}\right)
         = \diffp{p}{x} + \mu \left(\diffp[2]{u}{x} + \diffp[2]{u}{y}\right)
\end{equation}

\begin{equation}\label{eq:Q4NS2}
        \rho\left(u\diffp{v}{x} + v\diffp{v}{y}\right)
         = \rho g_y - \diffp{p}{y} + \mu \left(\diffp[2]{v}{x} + \diffp[2]{v}{y}\right)
\end{equation}

\begin{equation}\label{eq:MassConservationQ4}
    \diffp{u}{x} + \diffp{v}{y} = 0
\end{equation}

Conforme as equações \eqref{eq:Q4NS1}, \eqref{eq:Q4NS2} e \eqref{eq:MassConservationQ4},
temos que as velocidades horizontal $u$ e vertical $v$ são funções de duas variáveis: $x$ e $y$.
A injeção de ar pela face sul é de baixo para cima, logo temos efeito da gravidade no fluido,
representada pelo termo $\rho g_y$ em \eqref{eq:Q4NS2}. 

Os gradientes de pressão na direçao $x$ e $y$
não foram informados no enunciado do problema, mas podemos assumir que existem gradientes 
causando os movimentos do fluido em ambas direçoes $x$ e $y$. Nessa interpretação, usamos a analogia
que gradientes de pressão causam movimento de fluidos da mesma forma que gradientes de potencial
elétrico (tensão) geram fluxo de corrente elétrica. 

Uma vez identificado as equações diferenciais que modelam o problema, temos que determinar
as condições de contorno. Na face oeste, temos

\begin{equation}\label{eq:q4WestBoundary}
    \begin{cases}
        u(0, y) = 0.06 \un{m/s}, & 0 \leq y \leq 1 \un{cm} \\
        v(0, y) = 0.06 \un{m/s}, & 0 \leq y \leq 1 \un{cm} \\
    \end{cases}
\end{equation}

Na face leste temos sáida, que significa escoamento plenamente
desenvolvido, ou seja, derivadas espaciais nulas.

\begin{equation}\label{eq:q4EastBoundary}
    \begin{cases}
        \diffp{u}{x} = \diffp{u}{y} = 0, & x = 20 \un{cm}, 0 \leq y \leq 1 \un{cm}  \\
        \diffp{v}{x} = \diffp{v}{y} = 0, & x = 20 \un{cm}, 0 \leq y \leq 1 \un{cm}  \un{cm}\\
    \end{cases}
\end{equation}

Na face sul temos uma região da parede que é fixa, e outra que é porosa com injeção
de ar. Assim, temos

\begin{equation}\label{eq:q4SouthBoundary}
    \begin{cases}
        u(x, 0) = 0, & 0 \leq x \leq 20 \un{cm}  \\
        v(x, 0) = 0, & 0 \leq x \leq 10 \un{cm}  \\
        v(x, 0) = 0.02 \un{m/s}, & 10 \leq x \leq 20 \un{cm}  \\
    \end{cases}
\end{equation}

Na face norte temos condições de contorno simétricas à face sul

\begin{equation}\label{eq:q4NorthBoundary}
    \begin{cases}
        u(x, 1 \un{cm}) = 0, & 0 \leq x \leq 20 \un{cm}  \\
        v(x, 1 \un{cm}) = 0, & 0 \leq x \leq 10 \un{cm}  \\
        v(x, 1 \un{cm}) = 0.02 \un{m/s}, & 10 \leq x \leq 20 \un{cm}  \\
    \end{cases}
\end{equation}

Finalmente, o problema é modelado matematicamente pelas equações diferenciais parciais
\eqref{eq:Q4NS1}, \eqref{eq:Q4NS2} e \eqref{eq:MassConservationQ4}, com as condições de 
contorno especificadas em \eqref{eq:q4WestBoundary}, \eqref{eq:q4EastBoundary}, 
\eqref{eq:q4SouthBoundary} e \eqref{eq:q4NorthBoundary}.

\subsection{Análise Numérica}

Ao contrário dos Problemas 1, 2 e 3, discretizar as equações diferenciais \eqref{eq:Q4NS1}, \eqref{eq:Q4NS2} e \eqref{eq:MassConservationQ4}
e desenvolver um programa para analisar numericamente seria bastante trabalhoso. Assim, 
usamos o software gratuito CFD Studio para obter a distribuição de velocidades. 

As configurações usadas no software foram as seguintes:

\begin{itemize}
    \item Tipo de malha utilizado: malha cartesiana, como 10 divisões na vertical e 20 na horizontal
    \item Esquema de interpolação:
    \item Algoritmo de acoplamento pressão velocidade (P-V): 
    \item Coeficiente de sub-relaxação da pressão:
    \item Pressão de referência
    \item Local de imposição de pressão de referência
    \item Número de iterações necessárias até convergir
    \item Solver utilizado
\end{itemize}




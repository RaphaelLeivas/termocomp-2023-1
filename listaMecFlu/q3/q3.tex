
\section{Terceira Questão (9 pts)}

\subsection{Análise Analítica}

\numberwithin{equation}{section}
\numberwithin{figure}{section}

Para modelar o problema dado, temos que partir das equações de Navier-Stokes e 
da equação de conservação da massa. Assumindo escoamento incompressível (massa específica constante),
temos

\begin{equation}\label{eq:Q3NS1}
    \begin{split}
        \rho\left(\diffp{u}{t} + u\diffp{u}{x} + v\diffp{u}{y}+ w\diffp{u}{z}\right) \\
         = \rho g_x - \diffp{p}{x} + \mu \left(\diffp[2]{u}{x} + \diffp[2]{u}{y} + \diffp[2]{u}{z}\right)
    \end{split}
\end{equation}

\begin{equation}\label{eq:Q3NS2}
    \begin{split}
        \rho\left(\diffp{v}{t} + u\diffp{v}{x} + v\diffp{v}{y}+ w\diffp{v}{z}\right) \\
         = \rho g_y - \diffp{p}{y} + \mu \left(\diffp[2]{v}{x} + \diffp[2]{v}{y} + \diffp[2]{v}{z}\right)
    \end{split}
\end{equation}

\begin{equation}\label{eq:Q3NS3}
    \begin{split}
        \rho\left(\diffp{w}{t} + u\diffp{w}{x} + v\diffp{w}{y}+ w\diffp{w}{z}\right) \\
         = \rho g_z - \diffp{p}{z} + \mu \left(\diffp[2]{w}{x} + \diffp[2]{w}{y} + \diffp[2]{w}{z}\right)
    \end{split}
\end{equation}

\begin{equation}\label{eq:MassConservation}
    \nabla \cdot V = 0 \logo \diffp{u}{x} + \diffp{v}{y} + \diffp{w}{z} = 0
\end{equation}

As equações \eqref{eq:Q3NS1}, \eqref{eq:Q3NS2} e \eqref{eq:Q3NS3} são as três 
equações de Navier-Stokes, e a equação \eqref{eq:MassConservation} é a equação de conservação
da massa.

Com base nos dados do problema, podemos fazer várias simplificações nas equações acima:

\begin{itemize}
    \item Escoamento unidimensional apenas na direção $x$: $v = w = 0$
    \item Escoamento em regime permanente: $\diffp{u}{t} = 0$
    \item Gravidade atua apenas na componente $y$: $g_x = g_z = 0$
    \item Escoamento plenamente desenvolvido: $\diffp{u}{x} = \diffp[2]{u}{x} = 0$
    \item Na geometria analisada, o sistema de coordenadas possui apenas eixos $x$ e $y$: 
    $\diffp{u}{z} = \diffp[2]{u}{z} = 0$ 
\end{itemize}

Com todas essas simplificações, as equações acima se reduzem a 

\begin{equation}\label{eq:q3modelo}
    \diffp[2]{u}{y} - \diff{p}{x} = 0
\end{equation}

\noindent em que $u$ é função de apenas uma variável $y$.
Assim, temos as seguintes condições de contorno para \eqref{eq:q3modelo}:

\begin{equation}\label{eq:q3modeloContorno}
    \begin{cases}
        u(-h) = 0, \\
        u(h) = U,   
    \end{cases}
\end{equation}

De modo semelhante ao feito na Questão 3, podemos tentar obter uma solução analítica
para o modelo através de integração direta, assumindo que o gradiente de pressão 
na direção do escoamento $\diff{p}{x}$ é constante.
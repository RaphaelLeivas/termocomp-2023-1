
\section{Segunda Questão (9 pts)}

\subsection{Análise Analítica}

\numberwithin{equation}{section}
\numberwithin{figure}{section}

O escoamento de um fluido de viscosidade $\mu$ e massa específica $\rho$,
plenamente desenvolvido em regime permanente, em coordenadas cilíndricas,
é modelado pela equação diferencial 

\[ -\frac{1}{\rho} \diff{p}{z} + \frac{\mu}{\rho} \frac{1}{r} \frac{d}{dr} \left(r\diff{u}{r}\right) = 0 \]

\noindent note que podemos cancelar a massa específica $\rho$, obtendo

\begin{equation}\label{eq:modeloQ2}
    \frac{1}{r} \frac{d}{dr} \left(r\diff{u}{r}\right) = \diff{p}{z} \frac{1}{\mu}
\end{equation}

Com um gradiente de pressão constante, temos que a velocidade axial $u$ 
é uma função de apenas uma variável $r$. Assim, temos as condições de contorno:

\begin{equation}\label{eq:modeloQ2Contorno}
    \begin{cases}
        u(0) = \un{simetria} \logo \diff{u}{r} = 0,\\
        u(R) = 0 \\
    \end{cases}
\end{equation}

\noindent onde $R$ é o raio da tubulação por onde o fluido está escoando. Quando 
$r = R$ estamos em contato com as paredes da tubulação, que naturalmente têm 
velocidade nula. A primeira condição de contorno (simetria) é consequência do sistema de coordenadas
cilíndricas adotado.

Como \eqref{eq:modeloQ2} é uma EDO de primeira ordem, podemos tentar obter uma solução 
analítica por integração direta. 

\[ \int \frac{d}{dr} \left(r\diff{u}{r}\right) \, dr = \int r \diff{p}{z} \frac{1}{\mu} \, dr \]

\[ r \diff{u}{r} = \frac{r^2}{2} \diff{p}{z} \frac{1}{\mu} + C_1 \]

\[ \int \diff{u}{r} \, dr = \int  \frac{r}{2} \diff{p}{z} \frac{1}{\mu} + \frac{C_1}{r} \, dr \]

\begin{equation}\label{eq:Q2AnaliticGeral}
    u(r) = \frac{r^2}{4} \diff{p}{z} \frac{1}{\mu} + C_1 ln (r) + C_2
\end{equation}

A equação \eqref{eq:Q2AnaliticGeral} é a solução geral para a EDO \eqref{eq:modeloQ2}.
Para que ela atenda às condições de contorno de \eqref{eq:modeloQ2Contorno}, de imediato
temos $C_1 = 0$, pois o logaritmo natural não está definido para $r = 0$,
sendo que fisicamente já sabemos que há uma velocidade no centro da tubulação. Assim,
\eqref{eq:Q2AnaliticGeral} se reduz a 

\[ u(r) = \frac{r^2}{4} \diff{p}{z} \frac{1}{\mu} + C_2 \]

A constante $C_2$ pode ser identificada usando a condição de contorno $u(R) = 0$:

\[ 0 = \frac{R^2}{4} \diff{p}{z} \frac{1}{\mu} + C_2 \]

\[ C_2 = - \frac{R^2}{4} \diff{p}{z} \frac{1}{\mu} \]

Assim, a solução analítica do problema é 

\[ u(r) = \frac{r^2}{4} \diff{p}{z} \frac{1}{\mu} - \frac{R^2}{4} \diff{p}{z} \frac{1}{\mu}  \]

\begin{equation}\label{eq:Q2Analitic}
    u(r) = \frac{1}{4} \diff{p}{z} \frac{1}{\mu} \left(r^2 - R^2\right) 
\end{equation}

No problema temos os seguintes parâmetros:

\begin{itemize}
    \item Raio do tubo: $R=25 \un{mm}$
    \item Viscosidade: $\mu=1.01 \cdot 10^{-3} \un{ kg m$^{-1}$ s$^{-1}$}$
    \item Gradiente de pressão: $\diff{p}{z}=-12.928 \un{N/m}$
    \item Massa específica: $\rho=998 \un{kg m$^{-3}$}$
\end{itemize}

Com esses parâmetros, a Figura \ref*{fig:graficoAnaliticoQ2} mostra o perfil
de velocidade dentro da tubulação conforme a solução analítica em
\eqref{eq:Q2Analitic}.

\begin{figure}[h!]
    \caption{Solução analítica da questão 2.}
    \label{fig:graficoAnaliticoQ2}
    \centering
    \centerline{\includegraphics[scale=0.5]{graficoAnaliticoQ2.png}}
    \par{Fonte: elaboração própria.}
\end{figure}

O perfil de velocidade da Figura \ref*{fig:graficoAnaliticoQ2} é condizente com o 
esperado fisicamente: a velocidade é máxima no centro do tubo, e zero nas paredes na tubulação,
atendendo também às condições de contorno de \eqref{eq:modeloQ2Contorno}.
Além disso, igual visto durante as aulas da disciplina, temos um perfil parabólico para o 
escoamento plenamente desenvolvido na tubulação.

\subsection{Análise Numérica}

Concluída a análise analítica, podemos realizar uma análise numérica do problema. De modo semelhante
ao feito na Questão 1, o primeiro passo é discretizar a equação diferencial \eqref{eq:modeloQ2} via 
diferenças finitas.


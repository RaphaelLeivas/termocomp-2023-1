\section*{Anexo B}

Neste anexo está o código completo desenvolvido em R na questão 2. O código usado para gerar os gráficos não está exibido.

\begin{lstlisting}
    dpdz <- -12.928
    mu <- 1.01 * 10^-3
    R <- 25 * 10^-3
    N <- 50 # numero de pontos analisados 
    dr <- R/N # passo no espaco

    r_seq <- seq(from = 0.0, to = R, by = dr)

    N <- N + 1 # ajuste posicao zero

    A <- matrix(0, ncol = N, nrow = N)
    line <- 0
    
    # matriz coluna (vetor vertical) que representa os termos independentes
    b <- rep(0, N)
    
    for (ri in r_seq) {
      line <- line + 1
      
      if (ri == 0) {
        # condicao de simetria
        A[line, line + 1] <- 1
        A[line, line] <- -1
        
        b[line] <- (dpdz / mu) * (1/4) * dr^2
      } else if (ri == R) {
        # na borda do cano, velocidade nula (cond de contorno)
        # assim temos 1 * u_R = 0 ==> u_R = 0
        
        A[line, line] <- 1
        b[line] <- 0
      } else {
        # no interno
        A[line, line + 1] <- 1 + ri / dr
        A[line, line] <- -1 - 2 * ri / dr
        A[line, line - 1] <- ri / dr
        
        b[line] <- (dpdz / mu) * ri * dr
      }
    }
    
    u_numerical <- solve(A) %*% b
\end{lstlisting}
\section*{Anexo C}

Neste anexo está o código completo desenvolvido em R na questão 3. O código usado para gerar os gráficos não está exibido.

\begin{lstlisting}
  dpdx <- -178800
  mu <- 1.49
  h <- 5 * 10^-3
  U <- 1
  N <- 20 # numero de pontos analisados 
  dy <- 2 * h/N # passo no espaco

  N <- N + 1 # ajuste posicao incial

  y_seq <- seq(from = -h, to = h, by = dy)

  A <- matrix(0, ncol = N, nrow = N)
  b <- rep(0, N)
  
  line <- 0
  
  for (yi in y_seq) {
    line <- line + 1
    
    if (yi == -h) {
      # condicao de contorno placa de baixo
      # u0 = 0 ==> 1 * u0 = 0
      
      A[line, line] <- 1
      b[line] <- 0
    } else if (yi == h) {
      # condicao de contorno placa de cima
      # uN = U ==> 1 * uN = U
      
      A[line, line] <- 1
      b[line] <- U
    } else {
      # no interno, usa  a equacao de diferencas finitas deduzida
      A[line, line + 1] <- 1
      A[line, line] <- -2
      A[line, line - 1] <- 1
      
      b[line] <- (dpdx / mu) * dy^2
    }
  }
  
  u_numerical <- solve(A) %*% b
\end{lstlisting}
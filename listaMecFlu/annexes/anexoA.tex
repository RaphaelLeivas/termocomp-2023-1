\section*{Anexo A}

Aqui está o código completo desenvolvido em R na questão 1. O código de formatação dos dados com o pacote
\verb|flextable| não está exibido. 

\begin{lstlisting}
    U0 <- 1
    mu <- 0.29
    rho <- 891
    
    H <- 10 * 10^-2 # tamanho vertical do dominio, em metros
    T <- 1 # periodo total de analise, em segundos
    N <- 40 # numero de malhas (intervalos) da direcao y, unidimensional
    M <- 200 # numeros de pontos no tempo
    dy <- H / N # metros por divisao
    dt <- T / M # segundos por divisao
    K = (mu / rho) * dt / (dy^2) # deve ser que menor que 0.5
    
    N <- N + 1 # ajuste para incluir a posicao zero
    
    y <- c()
    
    # configura a malha
    for (yi in 1:N) {
      y[yi] <- (yi - 1) * dy
    }
    
    solutionsList <- list()
    
    boundaryVelBottom <- 1
    boundaryVelTop <- 0
    initialVel <- 0 # inicialmente em repouso
    
    initialMesh <- c()
    
    # associa condicao inicial a malha
    for (yi in 1:N) {
      if (yi == 1) {
        # se esta na base
        initialMesh <- append(initialMesh, boundaryVelBottom)
      } else if (yi == N) {
        # se esta no topo
        initialMesh <- append(initialMesh, boundaryVelTop)
      } else {
        # se nao, eh malha interna 
        initialMesh <- append(initialMesh, initialVel)
      }
    }
    
    solutionsList[[1]] <- initialMesh
    
    for (t in 1:M) {
      prevU <- solutionsList[[t]]
      
      newU <- c()
      for (yi in 1:N) {
        # se esta na base ou no topo, temos condicao de contorno
        if (yi == 1) {
          # se esta na base
          newU <- append(newU, boundaryVelBottom)
        } else if (yi == N) {
          # se esta no topo
          newU <- append(newU, boundaryVelTop)
        } else {
          # se nao, eh malha interna e usa a expressao deduzida
          newU <- append(newU, prevU[yi] + K * (prevU[yi + 1] - 2 * prevU[yi] + prevU[yi - 1] ))
        }
      }
      
      solutionsList[[t + 1]] <- newU
    }
    
\end{lstlisting}